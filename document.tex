% !TEX root = ./document.tex

\documentclass[a4paper, spanish]{article}

\usepackage[utf8]{inputenc}
\usepackage[spanish, es-tabla]{babel}
\usepackage[margin={10mm, 22mm}]{geometry}
\usepackage[T1]{fontenc}
\usepackage{amsmath}
\usepackage{amssymb}
\usepackage{dirtytalk}


\renewcommand*{\arraystretch}{1.7}

\newcommand{\norm}[1]{\left\lVert#1\right\rVert}

\title{Descomposición en Valores Singulares: Ejercicios a mano}
\author{Sergio García Prado \\ \texttt{sergio@garciparedes.me}}
\date{\today}
\begin{document}

  \maketitle

  \begin{align}
    A_{mxn} &= U_{mxm} \cdot \Sigma_{mxn} \cdot V_{nxn}^T
  \end{align}

  \section{Ejercicio 1}

  \begin{equation}
    A =
    \begin{bmatrix}
      1 & 1\\
      1 & 2\\
      2 & 4
    \end{bmatrix}
  \end{equation}

  \begin{align}
    A^T A &=
    \begin{bmatrix}
      1 & 1 & 2\\
      1 & 2 & 4
    \end{bmatrix}
    \cdot
    \begin{bmatrix}
      1 & 1\\
      1 & 2\\
      2 & 4
    \end{bmatrix}
    =
    \begin{bmatrix}
      6 & 11\\
      11 & 21
    \end{bmatrix} \\
    A A^T &=
    \begin{bmatrix}
      1 & 1\\
      1 & 2\\
      2 & 4
    \end{bmatrix}
    \cdot
    \begin{bmatrix}
      1 & 1 & 2\\
      1 & 2 & 4
    \end{bmatrix}
    =
    \begin{bmatrix}
      2 & 3 & 6 \\
      2 & 5 & 10 \\
      6 & 10 & 20
    \end{bmatrix} \\
  \end{align}


  \begin{align}
    det(A^T A - \lambda I)
    = \begin{vmatrix}
      6 - \lambda & 11 \\
      11 & 21 - \lambda
    \end{vmatrix}
    = \lambda ^ 2 - 27 \lambda + 5
  \end{align}


  \begin{align}
    det(A^T A - \lambda I) = 0
    \Rightarrow
    \lambda =
    \begin{cases}
    \frac{27 - \sqrt{709}}{2} \\
    \frac{27 + \sqrt{709}}{2}
    \end{cases}
  \end{align}



  \begin{align}
    \Sigma =
    \begin{bmatrix}
      \sigma_1 & 0 \\
      0 & \sigma_2 \\
      0 & 0
    \end{bmatrix}
    =
    \begin{bmatrix}
      \sqrt{\lambda_1} & 0 \\
      0 & \sqrt{\lambda_2} \\
      0 & 0
    \end{bmatrix} =
    \begin{bmatrix}
      \sqrt{\frac{27 - \sqrt{709}}{2}} & 0 \\
      0 & \sqrt{\frac{27 + \sqrt{709}}{2}} \\
      0 & 0
    \end{bmatrix}
  \end{align}


  \begin{align}
    \lambda_1 = \frac{27 - \sqrt{709}}{2} &\Rightarrow
    \begin{bmatrix}
      6 - \frac{27 - \sqrt{709}}{2} & 11\\
      11 & 21 - \frac{27 - \sqrt{709}}{2}
    \end{bmatrix}
    \cdot
    \begin{bmatrix}
      v_{11}^* \\
      v_{12}^*
    \end{bmatrix}
    =
    \begin{bmatrix}
      0 \\
      0
    \end{bmatrix}
    \Rightarrow
    v_1^*
    \begin{bmatrix}
      1 \\
      \frac{15 + \sqrt{709}}{22}
    \end{bmatrix} \\
    \lambda_2 = \frac{27 + \sqrt{709}}{2} &\Rightarrow
    \begin{bmatrix}
      6 - \frac{27 + \sqrt{709}}{2} & 11\\
      11 & 21 - \frac{27 + \sqrt{709}}{2}
    \end{bmatrix}
    \cdot
    \begin{bmatrix}
      v_{21}^* \\
      v_{22}^*
    \end{bmatrix}
    =
    \begin{bmatrix}
      0 \\
      0
    \end{bmatrix}
    \Rightarrow
    v_2^*
    \begin{bmatrix}
      1 \\
      \frac{15 - \sqrt{709}}{22}
    \end{bmatrix}
  \end{align}


  \begin{align}
    V^* =
    \begin{bmatrix}
      v_1^* & v_2^*
    \end{bmatrix} =
    \begin{bmatrix}
      1 & 1 \\
      \frac{15 + \sqrt{709}}{22}  & \frac{15 - \sqrt{709}}{22}
    \end{bmatrix}
  \end{align}

    \begin{align}
      v_1 &= \frac{v_1^*}{\norm{v_1^*}} =
      \begin{bmatrix}
        \frac{1}{\sqrt{1 + \frac{(15 + \sqrt{709})^2}{484}}} \\
        \frac{15 + \sqrt{709}}{22\sqrt{1 + \frac{(15 + \sqrt{709})^2}{484}}}
      \end{bmatrix} \\
      v_2 &= \frac{v_2^* - \langle v_2^*,v_1 \rangle v_1}{\norm{v_2^* - \langle v_2^*,v_1 \rangle v_1}} =
      \begin{bmatrix}
        \frac{1}{\sqrt{1 + \frac{(- 15 + \sqrt{709})^2}{484}}} \\
        \frac{15 - \sqrt{709}}{22\sqrt{1 + \frac{(- 15 + \sqrt{709})^2}{484}}}
      \end{bmatrix} \\
    \end{align}


    \begin{align}
      V =
      \begin{bmatrix}
        v_1 & v_2
      \end{bmatrix} =
      \begin{bmatrix}
        \frac{1}{\sqrt{1 + \frac{(15 + \sqrt{709})^2}{484}}} & \frac{1}{\sqrt{1 + \frac{(- 15 + \sqrt{709})^2}{484}}}  \\
        \frac{15 + \sqrt{709}}{22\sqrt{1 + \frac{(15 + \sqrt{709})^2}{484}}} & \frac{15 - \sqrt{709}}{22\sqrt{1 + \frac{(- 15 + \sqrt{709})^2}{484}}}
      \end{bmatrix}
    \end{align}

    \begin{align}
      u_1 &=
      \frac{1}{\sigma_1}Av_1 =
      \frac{1}{\sqrt{\frac{27 - \sqrt{709}}{2}}}
      \begin{bmatrix}
        1 & 1\\
        1 & 2\\
        2 & 4
      \end{bmatrix}
      \cdot
      \begin{bmatrix}
        \frac{1}{\sqrt{1 + \frac{(15 + \sqrt{709})^2}{484}}} \\
        \frac{15 + \sqrt{709}}{22\sqrt{1 + \frac{(15 + \sqrt{709})^2}{484}}}
      \end{bmatrix}
      =
      \begin{bmatrix}
        \frac{-23-\sqrt{709}}{12} \\
        \frac{1}{2} \\
        1
      \end{bmatrix} \\
      u_2 &=
      \frac{1}{\sigma_2}Av_2 =
      \frac{1}{\sqrt{\frac{27 + \sqrt{709}}{2}}}
      \begin{bmatrix}
        1 & 1\\
        1 & 2\\
        2 & 4
      \end{bmatrix}
      \cdot
      \begin{bmatrix}
        \frac{1}{\sqrt{1 + \frac{(- 15 + \sqrt{709})^2}{484}}} \\
        \frac{15 - \sqrt{709}}{22\sqrt{1 + \frac{(- 15 + \sqrt{709})^2}{484}}}
      \end{bmatrix}
      =
      \begin{bmatrix}
        \frac{-23+\sqrt{709}}{12} \\
        \frac{1}{2} \\
        1
      \end{bmatrix}\\
      AA^T u_3 &= 0
      \Rightarrow
      \begin{bmatrix}
        2 & 3 & 6 \\
        2 & 5 & 10 \\
        6 & 10 & 20
      \end{bmatrix}
      \cdot
      \begin{bmatrix}
        u_{31} \\
        u_{32} \\
        u_{33}
      \end{bmatrix}
      =
      \begin{bmatrix}
        0 \\
        0 \\
        0
      \end{bmatrix}
      \Rightarrow
      u_3^* =
      \begin{bmatrix}
        0 \\
        -2 \\
        1
      \end{bmatrix}
      \Rightarrow
      u_3 =
      \begin{bmatrix}
        0 \\
        \frac{-2}{\sqrt{5}} \\
        \frac{1}{\sqrt{5}}
      \end{bmatrix}
    \end{align}

    \begin{align}
      U =
      \begin{bmatrix}
        u_1 & u_2 & u_3
      \end{bmatrix} =
      \begin{bmatrix}
        \frac{-23-\sqrt{709}}{12} & \frac{-23+\sqrt{709}}{12} & 0 \\
        \frac{1}{2} & \frac{1}{2} & \frac{-2}{\sqrt{5}} \\
        1 & 1 & \frac{1}{\sqrt{5}}
      \end{bmatrix}
    \end{align}


    \begin{align}
      A &=
      \begin{bmatrix}
        1 & 1\\
        1 & 2\\
        2 & 4
      \end{bmatrix} \\
      U &=
      \begin{bmatrix}
        \frac{-23-\sqrt{709}}{12} & \frac{-23+\sqrt{709}}{12} & 0 \\
        \frac{1}{2} & \frac{1}{2} & \frac{-2}{\sqrt{5}} \\
        1 & 1 & \frac{1}{\sqrt{5}}
      \end{bmatrix} \\
      \Sigma &=
      \begin{bmatrix}
        \sqrt{\frac{27 - \sqrt{709}}{2}} & 0 \\
        0 & \sqrt{\frac{27 + \sqrt{709}}{2}} \\
        0 & 0
      \end{bmatrix} \\
      V^T &=
      \begin{bmatrix}
        \frac{1}{\sqrt{1 + \frac{(15 + \sqrt{709})^2}{484}}} &  \frac{15 + \sqrt{709}}{22\sqrt{1 + \frac{(15 + \sqrt{709})^2}{484}}}  \\
        \frac{1}{\sqrt{1 + \frac{(- 15 + \sqrt{709})^2}{484}}} & \frac{15 - \sqrt{709}}{22\sqrt{1 + \frac{(- 15 + \sqrt{709})^2}{484}}}
      \end{bmatrix}
    \end{align}

    \begin{align}
      \begin{bmatrix}
        1 & 1\\
        1 & 2\\
        2 & 4
      \end{bmatrix}
      =
      \begin{bmatrix}
        \frac{-23-\sqrt{709}}{12} & \frac{-23+\sqrt{709}}{12} & 0 \\
        \frac{1}{2} & \frac{1}{2} & \frac{-2}{\sqrt{5}} \\
        1 & 1 & \frac{1}{\sqrt{5}}
      \end{bmatrix}
      \cdot
      \begin{bmatrix}
        \sqrt{\frac{27 - \sqrt{709}}{2}} & 0 \\
        0 & \sqrt{\frac{27 + \sqrt{709}}{2}} \\
        0 & 0
      \end{bmatrix}
      \cdot
      \begin{bmatrix}
        \frac{1}{\sqrt{1 + \frac{(15 + \sqrt{709})^2}{484}}} &  \frac{15 + \sqrt{709}}{22\sqrt{1 + \frac{(15 + \sqrt{709})^2}{484}}}  \\
        \frac{1}{\sqrt{1 + \frac{(- 15 + \sqrt{709})^2}{484}}} & \frac{15 - \sqrt{709}}{22\sqrt{1 + \frac{(- 15 + \sqrt{709})^2}{484}}}
      \end{bmatrix}
    \end{align}


  \section{Ejercicio 2}


  \begin{equation}
    A =
    \begin{bmatrix}
      1 & -1 & 2\\
      1 & 2 & -1\\
      2 & 1 & 1
    \end{bmatrix}
  \end{equation}


  \begin{equation}
    A^TA =
    \begin{bmatrix}
      6 & 3 & 3 \\
      3 & 6 & -3 \\
      3 & -3 & 6
    \end{bmatrix}
  \end{equation}

  \begin{equation}
    AA^T =
    \begin{bmatrix}
      6 & -3 & 3 \\
      -3 & 6 & 3 \\
      3 & 3 & 6
    \end{bmatrix}
  \end{equation}


  \begin{align}
    det(A^T A - \lambda I)
    = \begin{vmatrix}
      6-\lambda & 3 & 3 \\
      3 & 6-\lambda & -3 \\
      3 & -3 & 6-\lambda
    \end{vmatrix}
    = -\lambda^3 + 18 \lambda ^ 2 - 81 \lambda
  \end{align}


  \begin{align}
    det(A^T A - \lambda I) = 0
    \Rightarrow
    \lambda =
    \begin{cases}
    9 \quad (\text{multiplicidad doble}) \\
    0 \quad (\text{multiplicidad simple})
    \end{cases}
  \end{align}

  \begin{align}
    \Sigma =
    \begin{bmatrix}
      \sigma_1 & 0  & 0\\
      0 & \sigma_1  & 0 \\
      0 & 0 & \sigma_2
    \end{bmatrix}
    =
    \begin{bmatrix}
      \sqrt{\lambda_1} & 0 & 0 \\
      0 & \sqrt{\lambda_1} & 0\\
      0 & 0 & \sqrt{\lambda_2}
    \end{bmatrix} =
    \begin{bmatrix}
      3 & 0 & 0 \\
      0 & 3 & 0 \\
      0 & 0 & 0
    \end{bmatrix}
  \end{align}



  \begin{align}
    \lambda_1 = 3 &\Rightarrow
    \begin{bmatrix}
      6-3 & 3 & 3 \\
      3 & 6-3 & -3 \\
      3 & -3 & 6-3
    \end{bmatrix}
    \cdot
    \begin{bmatrix}
      v_{11}^* & v_{21}^*\\
      v_{12}^* & v_{23}^*\\
      v_{13}^* & v_{23}^*
    \end{bmatrix}
    =
    \begin{bmatrix}
      0 \\
      0 \\
      0
    \end{bmatrix}
    \Rightarrow
    v_1^* =
    \begin{bmatrix}
      1 \\
      0 \\
      1
    \end{bmatrix},
    v_2^* =
    \begin{bmatrix}
      1 \\
      1 \\
      0
    \end{bmatrix}\\
    \lambda_1 = 3 &\Rightarrow
    \begin{bmatrix}
      6 & 3 & 3 \\
      3 & 6 & -3 \\
      3 & -3 & 6
    \end{bmatrix}
    \cdot
    \begin{bmatrix}
      v_{31}^* \\
      v_{32}^* \\
      v_{33}^* \\
    \end{bmatrix}
    =
    \begin{bmatrix}
      0 \\
      0 \\
      0
    \end{bmatrix}
    \Rightarrow
    v_3^* =
    \begin{bmatrix}
      -1 \\
      1 \\
      1
    \end{bmatrix}
  \end{align}

  \begin{align}
    V^* =
    \begin{bmatrix}
      v_1^* & v_2^* & v_3^*
    \end{bmatrix} =
    \begin{bmatrix}
      1 & 1 & -1 \\
      0 & 1 & 1 \\
      1 & 0 & 1
    \end{bmatrix}
  \end{align}


  \begin{align}
    v_1 &= \frac{v_1^*}{\norm{v_1^*}} =
    \begin{bmatrix}
      \frac{1}{\sqrt{2}} \\
      0 \\
      \frac{1}{\sqrt{2}}
    \end{bmatrix} \\
    v_2 &= \frac{v_2^* - \langle v_2^*,v_1 \rangle v_1}{\norm{v_2^* - \langle v_2^*,v_1 \rangle v_1}} =
    \begin{bmatrix}
      \frac{1}{\sqrt{6}} \\
      \sqrt{\frac{2}{3}} \\
      \frac{-1}{\sqrt{6}}
    \end{bmatrix} \\
    v_3 &= \frac{v_3^* - \langle v_3^*,v_1 \rangle v_1 - \langle v_3^*,v_2 \rangle v_2}{\norm{v_3^* - \langle v_3^*,v_1 \rangle v_1 - \langle v_3^*,v_2 \rangle v_2}} =
    \begin{bmatrix}
      \frac{-1}{\sqrt{3}} \\
      \frac{1}{\sqrt{3}} \\
      \frac{1}{\sqrt{3}}
    \end{bmatrix} \\
  \end{align}


  \begin{align}
    V =
    \begin{bmatrix}
      v_1 & v_2 & v_3
    \end{bmatrix} =
    \begin{bmatrix}
      \frac{1}{\sqrt{2}} & \frac{1}{\sqrt{6}} & \frac{-1}{\sqrt{3}} \\
      0 & \sqrt{\frac{2}{3}} &\frac{1}{\sqrt{3}} \\
      \frac{1}{\sqrt{2}} & \frac{-1}{\sqrt{6}} & \frac{1}{\sqrt{3}} \\
    \end{bmatrix}
  \end{align}

  \begin{align}
    u_1 &=
    \frac{1}{\sigma_1}Av_1 =
    \frac{1}{3}
    \begin{bmatrix}
      1 & -1 & 2\\
      1 & 2 & -1\\
      2 & 1 & 1
    \end{bmatrix}
    \cdot
    \begin{bmatrix}
      \frac{1}{\sqrt{2}} \\
      0 \\
      \frac{1}{\sqrt{2}}
    \end{bmatrix}
    =
    \begin{bmatrix}
      \frac{1}{\sqrt{2}} \\
      0 \\
      \frac{1}{\sqrt{2}}
    \end{bmatrix} \\
    u_2 &=
    \frac{1}{\sigma_2}Av_2 =
    \frac{1}{3}
    \begin{bmatrix}
      1 & -1 & 2\\
      1 & 2 & -1\\
      2 & 1 & 1
    \end{bmatrix}
    \cdot
    \begin{bmatrix}
      \frac{1}{\sqrt{6}} \\
      \sqrt{\frac{2}{3}} \\
      \frac{-1}{\sqrt{6}}
    \end{bmatrix}
    =
    \begin{bmatrix}
      \frac{-1}{\sqrt{6}} \\
      \sqrt{\frac{2}{3}} \\
      \frac{1}{\sqrt{6}}
    \end{bmatrix}\\
    A A^T u_3 &= 0
    \Rightarrow
    \begin{bmatrix}
      6 & -3 & 3 \\
      -3 & 6 & 3 \\
      3 & 3 & 6
    \end{bmatrix}
    \cdot
    \begin{bmatrix}
      u_{31}^* \\
      u_{32}^* \\
      u_{33}^*
    \end{bmatrix}
    =
    \begin{bmatrix}
      0 \\
      0 \\
      0
    \end{bmatrix}
    \Rightarrow
    u_3^* =
    \begin{bmatrix}
      -1 \\
      1 \\
      1
    \end{bmatrix}
    \Rightarrow
    u_3 =
    \begin{bmatrix}
      \frac{-1}{\sqrt{3}} \\
      \frac{1}{\sqrt{3}} \\
      \frac{1}{\sqrt{3}}
    \end{bmatrix}
  \end{align}


  \begin{align}
    U =
    \begin{bmatrix}
      u_1 & u_2 & u_3
    \end{bmatrix} =
    \begin{bmatrix}
      \frac{1}{\sqrt{2}} & \frac{-1}{\sqrt{6}} & \frac{-1}{\sqrt{3}} \\
      0 & \sqrt{\frac{2}{3}} & \frac{1}{\sqrt{3}} \\
      \frac{1}{\sqrt{2}} & \frac{1}{\sqrt{6}} & \frac{1}{\sqrt{3}}
    \end{bmatrix}
  \end{align}


  \begin{align}
    A &=
    \begin{bmatrix}
      1 & -1 & 2\\
      1 & 2 & -1\\
      2 & 1 & 1
    \end{bmatrix} \\
    U &=
    \begin{bmatrix}
      \frac{1}{\sqrt{2}} & \frac{-1}{\sqrt{6}} & \frac{-1}{\sqrt{3}} \\
      0 & \sqrt{\frac{2}{3}} & \frac{1}{\sqrt{3}} \\
      \frac{1}{\sqrt{2}} & \frac{1}{\sqrt{6}} & \frac{1}{\sqrt{3}}
    \end{bmatrix} \\
    \Sigma &=
    \begin{bmatrix}
      3 & 0 & 0\\
      0 & 3 & 0\\
      0 & 0 & 0
    \end{bmatrix} \\
    V &=
    \begin{bmatrix}
      \frac{1}{\sqrt{2}} & \frac{1}{\sqrt{6}} & \frac{-1}{\sqrt{3}} \\
      0 & \sqrt{\frac{2}{3}} &\frac{1}{\sqrt{3}} \\
      \frac{1}{\sqrt{2}} & \frac{-1}{\sqrt{6}} & \frac{1}{\sqrt{3}} \\
    \end{bmatrix}
  \end{align}

  \begin{align}
    \begin{bmatrix}
      1 & -1 & 2\\
      1 & 2 & -1\\
      2 & 1 & 1
    \end{bmatrix}
    =
    \begin{bmatrix}
      \frac{1}{\sqrt{2}} & \frac{-1}{\sqrt{6}} & \frac{-1}{\sqrt{3}} \\
      0 & \sqrt{\frac{2}{3}} & \frac{1}{\sqrt{3}} \\
      \frac{1}{\sqrt{2}} & \frac{1}{\sqrt{6}} & \frac{1}{\sqrt{3}}
    \end{bmatrix}
    \cdot
    \begin{bmatrix}
      3 & 0 & 0\\
      0 & 3 & 0\\
      0 & 0 & 0
    \end{bmatrix}
    \cdot
    \begin{bmatrix}
      \frac{1}{\sqrt{2}} & 0 & \frac{1}{\sqrt{2}} \\
      \frac{1}{\sqrt{6}} & \sqrt{\frac{2}{3}} & \frac{-1}{\sqrt{6}} \\
      \frac{-1}{\sqrt{3}} & \frac{1}{\sqrt{3}} & \frac{1}{\sqrt{3}} \\
    \end{bmatrix}
  \end{align}


\end{document}
