% !TEX root = ./document.tex

\documentclass[a4paper, spanish]{article}

\usepackage[utf8]{inputenc}
\usepackage[spanish, es-tabla]{babel}
\usepackage[margin={10mm, 22mm}]{geometry}
\usepackage[T1]{fontenc}
\usepackage{amsmath}
\usepackage{amssymb}
\usepackage{dirtytalk}


\renewcommand*{\arraystretch}{1.7}

\newcommand{\norm}[1]{\left\lVert#1\right\rVert}

\title{Descomposición en Valores Singulares: Ejercicios a mano}
\author{Sergio García Prado \\ \texttt{sergio@garciparedes.me}}
\date{\today}
\begin{document}

  \maketitle

  \begin{align}
    A_{mxn} &= U_{mxm} \Sigma_{mxn} V_{nxn}^T
  \end{align}

  \section{Ejercicio 1}

  \begin{equation}
    A =
    \begin{bmatrix}
      1 & 1\\
      1 & 2\\
      2 & 4
    \end{bmatrix}
  \end{equation}

  \begin{align}
    A^T A &=
    \begin{bmatrix}
      1 & 1 & 2\\
      1 & 2 & 4
    \end{bmatrix}
    \cdot
    \begin{bmatrix}
      1 & 1\\
      1 & 2\\
      2 & 4
    \end{bmatrix}
    =
    \begin{bmatrix}
      6 & 11\\
      11 & 21
    \end{bmatrix} \\
    A A^T &=
    \begin{bmatrix}
      1 & 1\\
      1 & 2\\
      2 & 4
    \end{bmatrix}
    \cdot
    \begin{bmatrix}
      1 & 1 & 2\\
      1 & 2 & 4
    \end{bmatrix}
    =
    \begin{bmatrix}
      2 & 3 & 6 \\
      2 & 5 & 10 \\
      6 & 10 & 20
    \end{bmatrix} \\
  \end{align}


  \begin{align}
    det(A^T A - \lambda I)
    = \begin{vmatrix}
      6 - \lambda & 11 \\
      11 & 21 - \lambda
    \end{vmatrix}
    = (6 - \lambda) (21 - lambda) - 121
    = \lambda ^ 2 - 27 \lambda + 5
  \end{align}


  \begin{align}
    det(A^T A - \lambda I) = 0
    \Rightarrow
    \lambda =
    \begin{cases}
    \frac{27 - \sqrt{709}}{2} \\
    \frac{27 + \sqrt{709}}{2}
    \end{cases}
  \end{align}



  \begin{align}
    \Sigma =
    \begin{bmatrix}
      \sigma_1 & 0 \\
      0 & \sigma_2 \\
      0 & 0
    \end{bmatrix}
    =
    \begin{bmatrix}
      \sqrt{\lambda_1} & 0 \\
      0 & \sqrt{\lambda_2} \\
      0 & 0
    \end{bmatrix} =
    \begin{bmatrix}
      \sqrt{\frac{27 - \sqrt{709}}{2}} & 0 \\
      0 & \sqrt{\frac{27 + \sqrt{709}}{2}} \\
      0 & 0
    \end{bmatrix}
  \end{align}


  \begin{align}
    \lambda_1 = \frac{27 - \sqrt{709}}{2} &\Rightarrow
    \begin{bmatrix}
      6 - \frac{27 - \sqrt{709}}{2} & 11\\
      11 & 21 - \frac{27 - \sqrt{709}}{2}
    \end{bmatrix}
    \cdot
    \begin{bmatrix}
      x_1 \\
      x_2
    \end{bmatrix}
    =
    \begin{bmatrix}
      0 \\
      0
    \end{bmatrix}
    \Rightarrow
    \begin{bmatrix}
      q \\
      \frac{15 + \sqrt{709}}{22} q
    \end{bmatrix}
    \Rightarrow
    \begin{bmatrix}
      1 \\
      \frac{15 + \sqrt{709}}{22}
    \end{bmatrix} \\
    \lambda_2 = \frac{27 + \sqrt{709}}{2} &\Rightarrow
    \begin{bmatrix}
      6 - \frac{27 + \sqrt{709}}{2} & 11\\
      11 & 21 - \frac{27 + \sqrt{709}}{2}
    \end{bmatrix}
    \cdot
    \begin{bmatrix}
      x_1 \\
      x_2
    \end{bmatrix}
    =
    \begin{bmatrix}
      0 \\
      0
    \end{bmatrix}
    \Rightarrow
    \begin{bmatrix}
      q \\
      \frac{15 - \sqrt{709}}{22} q
    \end{bmatrix}
    \Rightarrow
    \begin{bmatrix}
      1 \\
      \frac{15 - \sqrt{709}}{22}
    \end{bmatrix}
  \end{align}


  \begin{align}
    V^* =
    \begin{bmatrix}
      v_1^* & v_2*
    \end{bmatrix} =
    \begin{bmatrix}
      1 & 1 \\
      \frac{15 + \sqrt{709}}{22}  & \frac{15 - \sqrt{709}}{22}
    \end{bmatrix}
  \end{align}

    \begin{align}
      v_1 &= \frac{v_1^*}{\norm{v_1^*}} =
      \begin{bmatrix}
        \frac{1}{\sqrt{1 + \frac{(15 + \sqrt{709})^2}{484}}} \\
        \frac{15 + \sqrt{709}}{22\sqrt{1 + \frac{(15 + \sqrt{709})^2}{484}}}
      \end{bmatrix} \\
      v_2 &= \frac{v_2^* - \langle v_2^*,v_1 \rangle v_1}{\norm{v_2^* - \langle v_2^*,v_1 \rangle v_1}} =
      \begin{bmatrix}
        \frac{1}{\sqrt{1 + \frac{(- 15 + \sqrt{709})^2}{484}}} \\
        \frac{15 - \sqrt{709}}{22\sqrt{1 + \frac{(- 15 + \sqrt{709})^2}{484}}}
      \end{bmatrix} \\
    \end{align}


    \begin{align}
      V =
      \begin{bmatrix}
        v_1 & v_2
      \end{bmatrix} =
      \begin{bmatrix}
        \frac{1}{\sqrt{1 + \frac{(15 + \sqrt{709})^2}{484}}} & \frac{1}{\sqrt{1 + \frac{(- 15 + \sqrt{709})^2}{484}}}  \\
        \frac{15 + \sqrt{709}}{22\sqrt{1 + \frac{(15 + \sqrt{709})^2}{484}}} & \frac{15 - \sqrt{709}}{22\sqrt{1 + \frac{(- 15 + \sqrt{709})^2}{484}}}
      \end{bmatrix}
    \end{align}

    \begin{align}
      u_1 &=
      \frac{1}{\sigma_1}Av_1 =
      \frac{1}{\sqrt{\frac{27 - \sqrt{709}}{2}}}
      \begin{bmatrix}
        1 & 1\\
        1 & 2\\
        2 & 4
      \end{bmatrix}
      \cdot
      \begin{bmatrix}
        \frac{1}{\sqrt{1 + \frac{(15 + \sqrt{709})^2}{484}}} \\
        \frac{15 + \sqrt{709}}{22\sqrt{1 + \frac{(15 + \sqrt{709})^2}{484}}}
      \end{bmatrix}
      =
      \begin{bmatrix}
        \frac{-23-\sqrt{709}}{12} \\
        \frac{1}{2} \\
        1
      \end{bmatrix} \\
      u_2 &=
      \frac{1}{\sigma_2}Av_2 =
      \frac{1}{\sqrt{\frac{27 + \sqrt{709}}{2}}}
      \begin{bmatrix}
        1 & 1\\
        1 & 2\\
        2 & 4
      \end{bmatrix}
      \cdot
      \begin{bmatrix}
        \frac{1}{\sqrt{1 + \frac{(- 15 + \sqrt{709})^2}{484}}} \\
        \frac{15 - \sqrt{709}}{22\sqrt{1 + \frac{(- 15 + \sqrt{709})^2}{484}}}
      \end{bmatrix}
      =
      \begin{bmatrix}
        \frac{-23+\sqrt{709}}{12} \\
        \frac{1}{2} \\
        1
      \end{bmatrix}\\
      A^T A u_3 &= 0
      \Rightarrow
      \begin{bmatrix}
        2 & 3 & 6 \\
        2 & 5 & 10 \\
        6 & 10 & 20
      \end{bmatrix}
      \cdot
      \begin{bmatrix}
        u_{31} \\
        u_{32} \\
        u_{33}
      \end{bmatrix}
      =
      \begin{bmatrix}
        0 \\
        0 \\
        0
      \end{bmatrix}
      \Rightarrow
      u_3 =
      \begin{bmatrix}
        0 \\
        -2 \\
        q
      \end{bmatrix}
      \Rightarrow
      \begin{bmatrix}
        0 \\
        -2 \\
        1
      \end{bmatrix}
    \end{align}

    \begin{align}
      U =
      \begin{bmatrix}
        u_1 & u_2 & u_3
      \end{bmatrix} =
      \begin{bmatrix}
        \frac{-23-\sqrt{709}}{12} & \frac{-23+\sqrt{709}}{12} & 0 \\
        \frac{1}{2} & \frac{1}{2} & - 2 \\
        1 & 1 & 1
      \end{bmatrix}
    \end{align}


    \begin{align}
      A &=
      \begin{bmatrix}
        1 & 1\\
        1 & 2\\
        2 & 4
      \end{bmatrix} \\
      U &=
      \begin{bmatrix}
        \frac{-23-\sqrt{709}}{12} & \frac{-23+\sqrt{709}}{12} & 0 \\
        \frac{1}{2} & \frac{1}{2} & - 2 \\
        1 & 1 & 1
      \end{bmatrix} \\
      \Sigma &=
      \begin{bmatrix}
        \sqrt{\frac{27 - \sqrt{709}}{2}} & 0 \\
        0 & \sqrt{\frac{27 + \sqrt{709}}{2}} \\
        0 & 0
      \end{bmatrix} \\
      V^T &=
      \begin{bmatrix}
        \frac{1}{\sqrt{1 + \frac{(15 + \sqrt{709})^2}{484}}} &  \frac{15 + \sqrt{709}}{22\sqrt{1 + \frac{(15 + \sqrt{709})^2}{484}}}  \\
        \frac{1}{\sqrt{1 + \frac{(- 15 + \sqrt{709})^2}{484}}} & \frac{15 - \sqrt{709}}{22\sqrt{1 + \frac{(- 15 + \sqrt{709})^2}{484}}}
      \end{bmatrix}
    \end{align}

    \begin{align}
      \begin{bmatrix}
        1 & 1\\
        1 & 2\\
        2 & 4
      \end{bmatrix}
      =
      \begin{bmatrix}
        \frac{-23-\sqrt{709}}{12} & \frac{-23+\sqrt{709}}{12} & 0 \\
        \frac{1}{2} & \frac{1}{2} & - 2 \\
        1 & 1 & 1
      \end{bmatrix}
      \cdot
      \begin{bmatrix}
        \sqrt{\frac{27 - \sqrt{709}}{2}} & 0 \\
        0 & \sqrt{\frac{27 + \sqrt{709}}{2}} \\
        0 & 0
      \end{bmatrix}
      \cdot
      \begin{bmatrix}
        \frac{1}{\sqrt{1 + \frac{(15 + \sqrt{709})^2}{484}}} &  \frac{15 + \sqrt{709}}{22\sqrt{1 + \frac{(15 + \sqrt{709})^2}{484}}}  \\
        \frac{1}{\sqrt{1 + \frac{(- 15 + \sqrt{709})^2}{484}}} & \frac{15 - \sqrt{709}}{22\sqrt{1 + \frac{(- 15 + \sqrt{709})^2}{484}}}
      \end{bmatrix}
    \end{align}


  \section{Ejercicio 2}


\end{document}
